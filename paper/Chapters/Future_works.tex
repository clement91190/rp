% Chapter Template

\chapter{Results} % Main chapter title

\label{Chapter 5} % Change X to a consecutive number; for referencing this chapter elsewhere, use \ref{ChapterX}

\lhead{Chapter 5. \emph{Results}} % Change X to a consecutive number; this is for the header on each page - perhaps a shortened title

\section{ Simulation reproductability}
    ADD HERE -> discussion on the on the reproductability of the simulation
            -> graph showing noise of the results (all the scores for constant parameters)

\section{ Creatures taking advantages of simulation bugs}
    -> discussion
    -> link to video

\section { comparaison }

-> graph for each couple (model, learning method)



% In the second part of the project, my focus will be on improving the learning algorithms to lead to better solutions. One of the main problem of the movements of these structures is that it optimizes a given paramter (speed, distance with a certain amount of energy, rotation velocity...) but it does not give any control on the structure. In Modular and Swarm Robotics, if we want to make use of such creatures, we need to be able to interact with them, to modify their behaviour. To do so, I want to focus my work on building models to represent how these robots will interact with the world and the orders they receive. Recent deep learning techniques are made possible with the growing accessible computation power and they showed good results on complex problems such as Computer Vision or Sound Recognition. Some of these techniques are good candidate in order to generate complex pattern to understand the structure, generate oscillation for the direct control of joints (Echo State Networks for instance \cite{jaeger2007echo}) and lead to the concept of self-awareness of the mechanical structure of a robot. 


% Chapter Template

\chapter{Control} % Main chapter title

\label{Chapter 2} % Change X to a consecutive number; for referencing this chapter elsewhere, use \ref{ChapterX}

\lhead{Chapter 2. \emph{Control}} % Change X to a consecutive number; this is for the header on each page - perhaps a shortened title

%
Once the simulation is fully implemented, the goal is to use it as a testbench for different ways of controlling the structure. The second part of the project will be fully consecrated on this part. A few tools have been implemented to test some of the algorithms that are used in such problems. 

\section{PID Control of the joints}

One of the problem to control accurately a joint is to adapt the command so that it can adapt to difficult situation. For instance, it is easier to walk in a swimmingpool than on the ground and this is why people recovering after an injury do aquatic training. For a joint, it can be easy to do a movement in the air, but the same movement is more difficult when touching the ground. One way to avoid this problem is to implement PID (Proportional Integral Derivative) controller. This kind of controller is used in a lot of different situation in the world of automation to control degrees of freedom. In or case, servomotors often integrate such loops to account for these changes of use. In the simulation, at each step, the command of each degree of freedom of the structure is calculated using such a PID controller. 


\begin{figure}[htbp]
    \centering
    \includegraphics[scale=0.3]{Figures/servomotor.png}
    \rule{35em}{0.5pt}
    \caption[Answer of a joint to a sinusoid command]{Answer of the joint to a sinusoid command with PID Control}
    \label{fig:Snake}
\end{figure}


A few issues come from introducing PID in this problem. The stability is one of them. The PID loops need to be fast enough to be stable, which means that reducing the time between two steps of calculation on the physical world. The other problem is more a biological concern. The main goal of this project is to find solutions that are biologically inspired instead of using a classical automation approach. Therefore though the use of PID is necessary in many robotics applications, we will try to avoid using them in generating oscillations. 

\section{Central Pattern Generator}

Central Pattern Generators (CPGs) are neural networks, that can generate oscillation for the control of the muscles of or body. They are the consequence and the cause of the paradigm of periodic movement in the locomation of animals. Different models have been implemented to represent CPGs. We can represent them as a graph of coupled oscillator, where each node influence the behavior of its neighbours. For a first implementation I choose to test the model of CPG followed at EPFL ( ## INCLUDE CITATION ##). The CPG Neural Network in that case, is a graph that follow the physical architecture of the robot, settting one node for each joint (hinge or vertebra) on the structure. The dynamic of the CPG is determined by a coupling weight matrix $w_{ij}$ a phase bias matrix between nodes $\varphi_{ij}$, setting the frequency of the different oscillator $\omega_i$ and the desired amplitude and offset of the oscillation. We can compute the angle using the following system of equation and an integration method (I used the Runge-Kutta method in this project)

\begin{equation*}
    \dot{\phi_i} = \omega_i + \sum{w_{ij} * r_j * sin(\phi_i - \phi_j - \varphi_{ij}) \tag{1}
\end{equation*}
\begin{equation*}
        \theta_i = x_i + r_i * cos(\phi_i)  \tag{2}
\end{equation*}
This two equation gives the angle of the oscillator ($\theta_i$) depending on the state variable of a node: $x_i$, $r_i$, $\phi_i$, that can be described respectively as the offset, the amplitude and the phase of the oscillator.

\begin{equation*}
    \acute{r_i} = ar(\frac ar (R_i - r_i) - \dot{r_i}) \tag{3}
\end{equation*}
\begin{equation*}
    \acute{x_i} = ax(\frac ax (X_i - x_i) - \dot{x_i}) \tag{4}
\end{equation*}

Equations (3) and (4) describe the dynamic of the amplitude and offset (a second order dynamic that converge to the desired values). This trick is to ensure continuity in the oscillations, even if some of the parameters of the oscillator change. $a_r$ and $a_x$ are gains to control the dynamic ($a_r = a_x = 20 rad/s$ ##CITATION## ). 


A modification of this model is possible to plug the measured value of the degrees of freedom. Instead of using the second order control loop on $\theta_i$ which is achieved with the PID, we can set this control on the phase. Thatway, if the joint has troubles achieving his movement, for instance when hitting the ground, the phase will be modified and the perturbation will have an impact on othe joints through equation (1).

Oneway to do so is to add a term in the equation (1), with $\dot{\theta_{reali}}$ the mesured angle velocity of the joint. 
\begin{equation*}
    \dot{\phi_i} = \omega_i + \sum{w_{ij} * r_j * sin(\phi_i - \phi_j - \varphi_{ij}) + a_{\phi} * \frac {\dot{\theta_{reali}} - \dot{\theta_i}} {r_i * sin (\phi_i)} \tag{1}
\end{equation*}

If we derive (2) we get: 
\begin{equation*}
    \dot{\theta_i} = \dot{x_i} + \dot{r_i} * cos(\phi_i) + r_i * sin(\phi_i) * \dot{\phi_i} \tag{2'}
\end{equation*}

By making the assumption that the dynamic of the amplitude and the offset is slow compared to the phase, we get
\begin{equation*}
    \dot{\theta_i} = r_i * sin(\phi_i) * \dot{\phi_i} \tag{2''}
\end{equation*}
Thatway, if we consider small variation of the phase, we can deduce an error term on $\dot{\phi_i}$ from the error on $\dot{\theta_i}$ given by $\frac {\dot{\theta_{reali}} - \dot{\theta_i}} {r_i * sin (\phi_i)}$ that we can control with a gain ($a_{\phi}$)


\section{Learning}

It is then possible to learn the parameters of the CPG to optimize the movement of the structure. Instead of having to find the value of the angles, the CPGs act as basis function for the angles, and thatway we reduce the space of research to a space with finite dimensions. 


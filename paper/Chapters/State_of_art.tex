% Chapter Template

\chapter{State of the Art} % Main chapter title

\label{Chapter 1} % Change X to a consecutive number; for referencing this chapter elsewhere, use \ref{ChapterX}

\lhead{Chapter 1. \emph{State of the Art}} % Change X to a consecutive number; this is for the header on each page - perhaps a shortened title

%----------------------------------------------------------------------------------------
%	SECTION 1
%----------------------------------------------------------------------------------------

%\section{State of the Art}

The first remarquable examples of modular robotics learning to evolve in a 3D simulated world is due to Karl's Creature in 1994 \cite{karl}. Since then Genetic Algorithms have proven to be very efficient to solve motions problem especially in trying to maximize on parameter (speed for instance) \cite{marbach} \cite{heinen} \cite{schaffer}. Some of those learning algorithm are also tested on physical implementation of complex  structures \cite{aibo} \cite{schaffer}. More recently new techniques using reinforcment learning \cite{christensen} showed interesting results. The M-blocks initiative at MIT \cite{mitcubes} tackles the mechanical engineering chalenges behind modular robotics, but a lot of challenges remain from a software perspective in order to use modular robotics in the future fields we expect it to have an impact one, such as space (or non-friendly human environments) exploration and construction, medicine, ... 

